\documentclass[11pt, a4paper, ngerman, twoside]{article}

%% Die folgende Anleitung gilt für Linux:
%%
%%  Die Übersetzung dieses Dokuments erfolgt mit (ggf. öfter aufrufen):
%%  pdflatex seminar.tex
%%
%%  Falls die Bibliographie geändert wurde, dann zusätzlich:
%%  bibtex seminar
%%
%%  Bitte darauf achten, dass pdflatex und bibtex keine Warnungen und
%%  keine Fehler melden.
%%
%%  Das Ergebnis kann man ansehen:
%%  acroread seminar.pdf
%%
%%  Die Quelldatei einer deutschen Rechtschreibprüfung unterziehen:
%%  ispell -d deutsch -T latin1 seminar.tex

%%  Zur Unterstuetzung von Umlauten:
%\usepackage{ngerman}
%%  Die Kodierung des Dokuments wird angegeben durch:
\usepackage[utf8]{inputenc}

\usepackage{amssymb,amsthm,graphicx}
\usepackage{hyperref} % important: load after float
\hypersetup{
	pdfproducer={pdfeTex 3.14159-1.30.6-2.2},
	colorlinks=false,
	pdfborder=0 0 0	% keine Box um die Links!
}

\oddsidemargin0mm
\evensidemargin0mm
\textwidth159.2mm
\topmargin0mm
\headheight0mm
\textheight43\baselineskip
\usepackage{url}
\urlstyle{rm}

%%  Die Seitennumerierung wird später von uns korrigiert.
\setcounter{page}{1}

%% Die Konfiguration eigener Header und Footer
%% ermöglicht beispielsweise das Paket
\usepackage{fancyhdr}
%% Die folgenden Zeilen konfigurieren das Paket für unsere Zwecke
\fancypagestyle{plain}{\fancyhf{}\fancyfoot[RO,LE]{\thepage}\renewcommand\headrulewidth{0pt}}
\fancyhf{}
\fancyhead[LE]{\myauthor}
\fancyhead[RO]{\mytitle}
\fancyfoot[RO,LE]{\thepage}
%% Hier setzen wir den Stil der Seiten des Dokuments so,
%% dass das von uns konfigurierte Paket fancyhr verwendet wird.
\pagestyle{fancy}

%%  Die Theorem-Umgebungen sind in amsthdoc.pdf beschrieben:
\renewcommand{\proofname}{Beweis}
\theoremstyle{plain}\newtheorem{Lemma}{Lemma}
\theoremstyle{plain}\newtheorem{Satz}[Lemma]{Satz}
\theoremstyle{definition}\newtheorem{Definition}[Lemma]{Definition}
\theoremstyle{definition}\newtheorem*{Beispiel}{Beispiel}
\theoremstyle{remark}\newtheorem*{Bemerkung}{Bemerkung}

%% Das Argument zu Date soll leer bleiben, 
%% damit dem Titel kein Datum hinzugefügt wird.
\date{}

% % % Ab hier sollte das Dokument bearbeitet werden! % % % 
%% Eigene \usepackage und \newcommand Befehle sind hier einzufügen.
%% Vor der Verwendung anderer Pakete muss eine Rücksprache mit dem Betreuer erfolgen.
% % %

\newcommand{\isdef}{=_{\mathrm{def}}}
\newcommand{\N}{\mathbb{N}}

\begin{document}
%%  Die Überschrift der Ausarbeitung ist zwischen {}
%%  in der nächsten Zeile einzutragen.
\title{Cognitive Questions in Software Visualization}
%%  Der Name des Autors oder die Namen der Autoren sind 
%%zwischen {} in der nächsten Zeile einzutragen.
\author{Frank Richter, Tamino Hartmann}

%%  Die nächste fünf Zeilen müssen unverändert bleiben!
\makeatletter
\let\mytitle\@title
\let\myauthor\@author
\makeatother
\maketitle
%% ab hier darf das Dokument wieder verändert werden!
\begin{abstract}

%%  Eine kurze Zusammenfassung der Ausarbeitung ist hier einzufügen.
%%  Sie sollte 4-12 Zeilen umfassen.
Das ist eine zu kurze Zusammenfassung.
Hier wird erläutert was in dieser Arbeit geleistet wird, aber ohne in Details zu gehen.
Nicht beschrieben wird \textit{wie} es gemacht wird, es sei denn gerade das ist interessant.

\end{abstract}

%%  Der Text der Ausarbeitung ist hier einzufügen.
%%  Er sollte 8-10 Seiten umfassen.
%%  Die Gliederung erfolgt mit \section und \subsection Befehlen.
%%  Referenzen innerhalb des Texts erfolgen mit \label und \ref Befehlen.
%%  Referenzen auf Literatur erfolgen mit dem \cite Befehl.
\section{Introduction}

Origin cite like this\cite{origin}.

\subsection{Overview}

good software visualisation isn’t simply a matter of mimicking paper-based tasks

the hope in visualisation is that expertise in presentation can compensate for inexpertness in readership

visualization aims to find simplicity in a complex artefact to produce selective representation of a complex abstraction (code)

unifying aim of software visualisation is to make information apparent

difficulty: represents information artefacts (as opposed to physical ones)

question of suitability

\subsection{Definition of Visualizations}

used to show different sorts of information:

presenting large data sets

demonstrating virtual machine

changing the perspective

display-based reasoning

distinction between visualizations: those that represent

data

code

algorithm

external structure versus cognitive structure

structure versus purpose or expectation

plan structure is how programmers conceptualise programs

at present there is a shortage of software visualizations explicitly built around cognitive representations

no idea how a cognitive representation would look like

define cognitive

\section{Reasons for using Software Visualizations}

space of intentions for software visualizations

what it is used for (experts view):

externalize images of thought

setting aside information

produce focus for knowledge transfer

to provide tools for thinking

harness computer as a collaborator in problem solving

intentions of software visualization for use by non-experts:

to provide glimpse of an expert’s imagery

to provide a visualization of expert’s reasoning process

to provide tool that are consistent with expert behavior

visualization of execution

provide additional information that non-experts find difficult to envision

assumption: non-experts have a less developed or less appropriate internal imagery

visualizations for non-experts often with instructive role

visualizations not useful without insight about their use

\section{Graphical Visualizations and their Aspects}

how to distinguish visualization software from viewing software?

essence of visualization:

alternative representation includes changes to

structure

modality

alternative emphasis include explicit representation of temporal behavior through

instantaneous animation

historical integration

spatial application domains can be better visualized with graphics (but both always possible)

growing evidence that graphical representations can be harder work and produce poorer performance than textual ones 

richness: graphical appear potentially richer than textual

gestalt

mapping to the domain

Accessibility and comprehensibility

fun: allows the programmer more freedom to play around with layout

illusion of accessibility may be more important than the reality

\section{Concepts of Visualizations}

\subsection{Tasks of Visualizations}

descriptions of visualization systems rarely specify any particular task that they are intended to support

cross-referencing strategy is important for experts

two axioms about information representation are:

finding and using information is easier when the form of the information sought has a cognitive fit

presentation highlights some kinds of information, likely obscures other kinds

\subsection{Information Dimensions in Visualizations}

Text is linear one-dimensional

Pictures are two dimensional

Moving pictures are three dimensional

Virtual reality adds a dimension (4d?)

eight variables can be utilised in a printed graphic image

each dimension comes at a cost

limited view 

economical line of research might be to discover how much can be achieved by two

the allocation of the plane should be the most careful decision taken

thinking less about building virtual reality systems and more about what to do with two dimensional display devices

\subsection{Influence of Personal Skill and Variation}

readership skills

visibility doesn’t guarantee effective interpretation

novice readers lack search an inspection strategies and get distracted by surface features

experts are better at recognising what is relevant and disregarding the irrelevant

visibility is in the eye of the beholder

graphical cultures: little of graphical culture is codified and some conventions vary among sub-cultures

presentation skills: guiding the reader by associating perceptual cues

cognitive style

experts have a tendency to create a visualization for a particular problem even if it will never be useful for another problem

software visualization is for coping with unknowns

other person’s visualization used seldom because the SV are made for single problem, and thus not applicable to current software

experts want from SV (comparable to programming languages) control, scale, speed, truth

\subsection{Multiple Representations}

SV usually presents two or more informations

usually one better

use of multiple representations:

used separately

different views on a conceptual space

bridging representations (one to understand the other)

heterogeneous inference (alternation between the two required to comprehend problem)

useful awkwardness (???)

\subsection{Influence of Conventions on Visualizations}

interpretation of visualization is influenced by general knowledge of graphical conventions

conflict between mathematical and metaphorical conventions

SV strongly depends on: (for widespread use)

cultural dependency

conflicting metaphors

commercial interests

unsupported intuitions

\section{Conclusion}

don’t assume too much

can’t highlight everything

do not presume upon your user’s contexts

SV can at present only be designed for one context

spontaneousness of programmers at solving problems ignored at own risk

%%  Der Text der Ausarbeitung ist hier zu Ende.
%%  Die nächste Zeile bleibt unverändert.
\bibliographystyle{plain}
\bibliography{seminar}

\end{document}
